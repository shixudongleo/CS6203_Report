\section{Introduction}
\label{sec:sec_intro}

The goal of lifelogging is to record and archive all information in one's life. 
This may include all texts, visual information, audio, media activity as well 
as biological data from sensors of one's body. The information would be archived
for the benefit of the lifelogger, and shared with others in various degrees as
controlled by him/her.  With the advances of technology, lifelog is becoming more
 and more popular. Fitness\footnote{http://www.fitbit.com/} tracker as one
 example is wildly worn by people to record  their sports activity, steps taken, distance traveled and calories
 burned.  Using this information, you can choose food, weight, and sleep
 tracking  to manage all aspects of your health.  At the same time, recent 
 technology has make wearable devices more accessible to common people. Google
 glass, and GoPro are just two brilliant example. With their support,  people
 are able to log their visual information.
Compares to other type of lifelog, vision-based lifelogs are more fruitful, and
if using properly, can help make people's life better. Recognizing objects of 
what people are watching can provide essential information about a person's 
activity and situation, thus have far-reaching impact on the application of 
vision in everyday life. The egocentric viewpoints from a wearable camera have
unique advantages in recognizing objects, such as a close view and seeing
objects in their natural positions.
With the vast amount of video and image collected daily, visual lifelog present 
challenges in analysis and store in a form with enable user to retrieved easily 
and extract useful information. 

%The goal of the project is an application that 
%analyses vision-based lifelogs stream and enable user to retrieve in low
% latency manner. The application would be focus on quantatively analyzing what people
%see,  such as; how frequent have one person meet people during a day?; how long 
%have one person been reading book?, and so on. Using these information, we
% would like to measure the similarity of people.

Objective:

Contribution. We analyses and proposes an approach for using
big data framework in video processing. 

This paper is organized as follows. Section 2 surveys related work on video
processing at large scale. Section 3 gives an overview of opportunities
and challenges when using big data framework for video processing. Section 4
describes our design of large scale video processing stack and our
implementation of PeopleCounting application as a case study. We evaluate
accuracy and efficiency of our implementation in Section 5, and conclude in
Section 6.
