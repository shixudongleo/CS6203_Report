\section{Related Work}
\label{sec:sec_background}


\TODO{overview}
\TODO{image processing \& big data framework}
\TODO{video processing \ldots }

There is recent work in the computer vision community that makes use of
MapReduce framework. 

\subsection{Image Processing and Big Data Framework}
\citeauthor{2009_ICME_Liu}~\cite{2009_ICME_Liu} proposed a face tracking
algorithm that uses multiple cues and a particle filtering algorithm. The mappers were applied in parallel
over the particle predictions and the reducers computed the updated parameters.
Experiments were carried out on a shared-memory implementation of MapReduce. 

\citeauthor{2009_ICCV_Li}~\cite{2009_ICCV_Li} developed a landmark
classification system that uses bag-of-feature vector and structured SVMs to classify landmarks visually iin
each photo in a user's photo-stream. They used a dataset of 6.5 million images
taken from Flickr and an experiments using MapReduce. Though the MapReduce
algorithm was not described, feature computation was mentioned to be the primary
bottleneck. 

\citeauthor{2009_WSMC_Kennedy}~\cite{2009_WSMC_Kennedy} explored a method to
generate image tags similar to those found in the ESP game while producing more specific tags. They used
MapReduce to directly search the nearest neighbor in their implementation. 19.6
million images on Flickr were used in their experiment. 

\citeauthor{2010_KDD_White}~\cite{2010_KDD_White} discussed both the high level
theory and the low level implementation for several computer vision algorithms:
classifier training, sliding windows, clustering, bag-of-features, background
subtraction and image registration. 

\citeauthor{2011_thesis_sweeney}~\cite{2011_thesis_sweeney} created a tool that
makes development of large-scale image processing and image-based vision projects
accessible. They empower developers to create large-scale image applications
with ease.

\subsection{Video Processing and Big Data Framework}
\citeauthor{2012_SMCCSE_Kim}~\cite{2012_SMCCSE_Kim} proposed a Hadoop-based
multimededia system for video transcoding processing on PaaS paltform. Their
system reduce the encoding time for transcoding large amounts of video files
into specific formats depending on user-requested options(such as resolution,
bit rate and frame rate). 

\citeauthor{2013_Pivotal}~\cite{2013_Pivotal} ingested video into Hadoop and
perform parallel and distributed transcodiing to create a sequence file format
containing JPEG images of the individual video format. Although they
demonstrated the framework only with MPEG-2 vidoe format as an example, it is
easy to imagine extending the framework to support a variety of format with the
help of corresponding decoders that can be called from Mappers. 

\TODO{summarize: most works focus on image processing, not much work on video
processing - only video transcoding}
 